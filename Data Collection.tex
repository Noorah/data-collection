%% LyX 2.1.2 created this file.  For more info, see http://www.lyx.org/.
%% Do not edit unless you really know what you are doing.
\documentclass[english]{article}
\usepackage[T1]{fontenc}
\usepackage[latin9]{inputenc}
\usepackage{textcomp}
\usepackage{graphicx}

\makeatletter

%%%%%%%%%%%%%%%%%%%%%%%%%%%%%% LyX specific LaTeX commands.
%% Because html converters don't know tabularnewline
\providecommand{\tabularnewline}{\\}

\makeatother

\usepackage{babel}
\begin{document}

\title{Data Collection}

\maketitle

\section*{Overview }

An embedded sensor architecture, was used to gather acoustic data
from ground vehicles at various locations around the Southampton University
area. The simplest form of acoustic data collection system consists
of an acoustic source and a collection circuit. The circuit consists
of a microphone, a preamplifier, an amplifier, connecting cables,
a recording system, and a power source. Microphones are used to convert
acoustic pressure fluctuations in the atmosphere into electrical signals.
The acoustic signals were pre-amplified with a selectable gain before
being passed onto the Raspberry Pi for more signal processing and
later analysed on a personal computer. The Data and Configuration
files are the two files that are saved on the Raspberry Pi, where
the former contains the actual data samples and the latter contains
configuration settings of a specific recording.

Sound sources have their own distinct characteristics to allow identification
and tracking of the source. However, as a sound propagates through
the atmosphere, it may be affected by a variety of meteorological
and atmospheric conditions. The sound received can also be modified
by surface conditions such as terrain and other geographical features.
Obstructions in the line of sight can also influence the received
amplitude and frequency of the signal although this may not be a major
requirement to receive signal. The characteristics of recording systems
can also influence received data therefore extra care was taken to
try to develop a reliable recording system. Planning to capture acoustic
measurements in the atmosphere requires that certain variables are
identified, evaluated and documented. At a minimum, these include
the characteristics of the sound source, a description of the recording
system, characterization of meteorological and ground conditions along
the propagation path, a description of the terrain, and notes on nuisance
sound sources.

Acoustic signatures are normally non-stationary {[}1{]} and are often
corrupted by propagation effects, noise, and interference from the
environment. The spectral characteristics of vehicle noise are distinctive
and their acoustic signatures are dominated by narrow band spectral
peaks. The main problem faced in classification is the selection of
proper feature vectors that will be stable and class specific.


\section{Setup}

The collection setup involved two directional microphones placed about
30cm apart to allow cross correction analysis of the acoustic signals.
The microphones used offer a simple low cost solution and provide
enough signal to noise ratio for vehicle sound recordings with low
sensitivity to wind effects. The microphones feed into an amplification
stage circuit before the signal is passed onto the ADC circuit. Signals
from the ADC stage are then passed to the raspberry pi for further
signal processing and data storage. The raspberry pi is powered by
a portable battery within the enclosure. The pi then connect to a
laptop using putty as a terminal interface to access and analyses
the input signals. A video feed of the respective vehicles is also
captured to help in the matching of different vehicles with their
acoustic signatures. The video is captured using a GoPro hero3 camera
and the data files from the raspberry pi are converted to wav files
using Matlab and stored on the laptop using FileZilla. The data collection
circuitry is all enclosed in a single casing that is placed at around
1.5 to 2m from the side of the road with microphones protruding out.
Figure 1 shows the acoustic data collection setup. Care is taken when
placing the casing on the ground surface as Seismic vibrations can
influence the signal output. The environment has a dramatic effect
on the transmission of acoustic signals through the atmosphere. Transmission
losses are frequency dependent, so the received signal depends upon
atmospheric conditions along the propagation path. The data log is
a time referenced spread sheet used to record source activity and
changes that take place during a test. Unusual conditions about the
test location are all recorded down in the Configuration file. The
position of the sensor in relation to the source such as the height,
direction and distance would affect the received signal\textquoteright s
characteristics therefore a flat frequency response microphone allows
collection of data with little modification due to the sensor\textquoteright s
characteristics. 

\begin{figure}
\includegraphics[scale=0.6]{\string"collection setup\string".eps}

\protect\caption{Acoustic Data Collection Setup}


\end{figure}



\section{Data Collection Strategy }

Data collection was performed by placing the collection circuit on
the side of the road. The road where both urban and suburban highways
with both single and dual traffic lanes. Having a directional microphone
is beneficial to collecting signals directly emitted from the moving
vehicles whilst minimising wind and other unwanted noise signals.
Initially the plan pertained to collecting sound recordings with limited
background noise to eliminate the need for filtering but this proved
hard to achieve as these conditions where had to obtain. To recognise
acoustic events efficiently, recordings of about 5-10 min from various
setup environments were recorded. To avoid data contamination from
Human sources I ensured to refrain from speaking or making necessary
noises when working around the acoustic sensors. After finding a suitable
site the equipment was placed on the ground and setup from a laptop
using the putty terminal interface. The setup involved resetting the
raspberry pi, checking that all the connections were intact and documenting
any details about location and system configuration.

The events in the recordings were manually annotated by specifying
the name and exact location of each audible event within the files.
Figure 2 provides sample database of the recordings made at Kitchener
road. Vehicles passing the microphone pair were extracted from raw
input data by listening to audio and the watching contemporary video
in order to gather information about vehicle type and speed. Suitable
vehicle categories were defined according to different class definitions
which later used to establish the categories for the classification
system. Four different class where been for this project\textquoteright s
classification system namely, car, truck, van and motorcycles. A large
number vehicles per category are needed to ensure reliable feature
extraction and classification performance. 

\begin{figure}
\includegraphics[scale=0.8]{\string"extracted events\string".eps}

\protect\caption{Number of events extracted for a samle context of the recording}


\end{figure}



\subsection{Sites }

The sites were selected to capture the sound of moving vehicles with
certain characteristics in mind such as locations with/without much
car acceleration or gear change, different types of vehicles likely
to be observed at a sites, different environmental conditions and
having multiple or a single car at a time. The recordings were attempted
without capturing undue attention from motorists and other road users.
Road test where made on areas around Southampton including Highfield
lane, St Denys etc. Table 1 shows a few of these area and their suitability
as testing site and Figure 3 illustrates the location of the sites
on google map. 

\begin{table}
\begin{tabular}{|c|c|c|c|}
\hline 
Location & Type & Road Speed limit & Comments\tabularnewline
\hline 
\hline 
Highfield lane & Urban dual lane & 40 & 12�C, cloudy, mild winds\tabularnewline
\hline 
Violet Road & Urban single lane & 30 & 10�C, sunny, moderate winds\tabularnewline
\hline 
St Denys Rd & Suburban highway & 50 & 15�C, sunny, mild winds\tabularnewline
\hline 
Portswood Rd & Urban dual lane & 40 & 9�C, cloudy, moderate winds\tabularnewline
\hline 
A27 & Suburban highway & 50 & 9�C, cloudy, moderate winds\tabularnewline
\hline 
Broadlands Rd & Urban single lane & 30 & 12�C, cloudy, mild winds\tabularnewline
\hline 
\end{tabular}

\protect\caption{A list of a few of the data collection sites}


\end{table}


\begin{figure}
\includegraphics[scale=0.7]{\string"collection sites\string".eps}

\protect\caption{Data collection sites illustrated on google maps}


\end{figure}



\subsection{Analysis }

Collected vehicle data was analysed and studied using Matlab. Matlab
is a multipurpose software tool that can be used for analysis of recorded
data. It was also used for labelling and observing the characteristics
of the recorded vehicle data as illustrated in Figure 4. Analysing
the energy contour showed that with the passing of each vehicle, there
was it was an increase in energy. The maximum energy is recorded at
the instant a vehicle crosses the audio recorder. The data file stored
on the raspberry pi was also converted to wave file using a dat2wav
Matlab script. The analysis of vehicle audio recordings from the wave
file indicated that detection of energy peaks may give an approximate
count of vehicles passing the recorder at a given point. Analysis
from initial phase of data collection showed that many of the recording
where affected by noise from the hardware. This noise was quite significant
rendering many of the initial recordings unusable. A more reliable
design hardware was later created that was used for phase 2, 3 and
4 of data collection. Data was mainly stored on a laptop with an external
backup and later uploaded to google drive to be accessed by the other
group members.

Vehicles of the same type and working in similar conditions normally
generate similar sound features, or noise signatures. When travelling
at different speeds, under different road conditions, or with different
acceleration, a vehicle emits different noise patterns. These noises
are sampled and grouped in a series of short time frames. When the
spectrum changes with time, it can be described in the frequency domain
as the change of frequency spectrum distribution over frames. When
a vehicle is moving it may experience bumps that also causes changes
in the frame's spectrum. For a car passing by, there can be more than
2 seconds of sustained signal available therefore a frame of about
0.2 second is used for each spectrum analysis making at least several
dozen samples available for classification.

\begin{figure}
\includegraphics[scale=0.6]{\string"signal representation\string".eps}

\protect\caption{Acoustic signal representation in the time domain using Matlab}


\end{figure}



\section{Risk assessment }

As part of managing the safety issues while out making recording,
a risk assessment was necessary to evaluate the likely hazards and
preventive methods needed to avoid accident on the road. A sample
of the risk assessment form is attached in the Appendix. It was necessary
to have a copy of this document while collecting data and presented
to any law enforcers. For some risks, particular control measures
where needed and this assessment helped in identifying where I needed
to look at certain risks and particular control measures to implement.
One of the most important aspects of the risk assessment was to accurately
identifying the main potential hazards. A good starting point was
to always walk around the collection site and think about any hazards
that could to lead injuring me or other road user. Among the safety
measures undertaken while collecting data involved wearing a high
visibility jacket and staying well clear of the road after the setup
is done and ensuring the sites have a curb large enough not to destruct
other road users. 
\begin{thebibliography}{1}
\bibitem{key-1} Automated Vehicle Detection and Classification using
Acoustic and Seismic Signals\end{thebibliography}

\end{document}
